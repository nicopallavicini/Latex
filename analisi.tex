\documentclass[11pt, a4paper]{article}
\usepackage[italian]{babel}
\usepackage{amsmath}
\author{Nicolò Pallavicini}
\title{Analisi Matematica 1}
\begin{document}
\maketitle
\newpage

\section{Insiemi Numerici}
\subsection{Irrazionalità di $\sqrt{2}$}
\subsubsection*{Dimostrazione}
Supponiamo per assurdo che $\sqrt{2}$ $\in$ Q. Allora se così fosse, $\sqrt{2}$ sarebbe rappresentabile come $\frac{n}{m}$ con n,m $\in{N}$ e primi tra loro. Allora risolvendo l'equazione avremmo: \begin{math}
2=\frac{n^2}{m^2},$\hspace{0.2cm}$2m^2=n^2
\end{math}.
\\
Ciò implica che m è pari, ma quindi anche n deve essere pari e può essere scritto come \begin{math} n=2k \end{math}
e quindi l'equazione può essere riscritta nella forma: \begin{math} 2m^2=4n^2 \end{math} abbiamo quindi trovato che i due numeri non sono ridotti a fattori primi. \textbf{Assurdo per ipotesi, $\sqrt{2}$ deve essere irrazionale.}

\subsection{Numerabilità di $\emph{Q}$}
\subsubsection*{Dimostrazione}
Dimostriamo che $ Q^+$ è numerabile. Rappresentiamo ogni numero razionale come $\frac{n}{m}$ con n,m interi positivi.\	
\\
\\
\begin{tabular}{|c|c|c|c|}
\hline
\rule [-0,3cm]{0mm}{0,7cm}
$\frac{1}{1}$ & & & \\
\hline
\rule [-0,3cm]{0mm}{0,7cm}
$\frac{1}{2}$ & $\frac{2}{1}$ & &\\
\hline
\rule [-0,3cm]{0mm}{0,7cm}
$\frac{1}{3}$ & $\frac{2}{2}$ & $\frac{3}{1}$ &\\
\hline
\rule [-0,3cm]{0mm}{0,7cm}
$\frac{1}{4}$ & $\frac{2}{3}$ & $\frac{3}{2}$ & $\frac{4}{1}$\\
\hline
... & ... & ... & ...\\
\hline
\end{tabular}
\\
\\
Ogni riga ha una lunghezza finita, quindi possiamo mettere in corrispondenza biunivoca con N.
\\
\\
\begin{tabular}{lcrlcrlcrl}
1 & 2 & 3 & 4 & 5 & 6 & 7 & 8 & 9 & ...\\
$\frac{1}{1}$ & $\frac{1}{2}$ & $\frac{2}{1}$ & $\frac{1}{3}$ & $\frac{3}{1}$ & $\frac{1}{4}$ & $\frac{2}{3}$ & $\frac{3}{2}$ & $\frac{4}{1}$ & ...
\end{tabular}
\\
\\
\textbf{L'insieme $Q^+$ è numerabile, quindi anche Q lo è.}

\subsection{Non Numerabilità di $\emph{R}$}
\subsubsection*{Dimostrazione}
Dimostriamo che l'intervallo [0,1] ha una cardinalità non numerabile da cui segue che R non è numerabile. Supponiamo per assurdo che [0,1] sia numerabile e disponiamo tutti i numeri reali dell'intervallo [0,1] in un elenco $r_1, r_2,r_3...$\\ Scriviamo ogni $r_i$ in forma decimale:
\begin{tabular}{c}
\begin{math} r_1=0,a_{11} a_{12} a_{13} ... \end{math}\\
\begin{math} r_1=0,a_{21} a_{22} a_{23} ... \end{math}\\
\begin{math} r_1=0,a_{31} a_{32} a_{33} ... \end{math}
\end{tabular}
\\
\\
\\
Definiamo $r=0,b_1 b_2 b_3 ...$ dove le cifre $b_i$ sono: 
$
\left \{
\begin{array}{l}
a_{ii}=0,1,2,3,4 \rightarrow b_i=5\\
a_{ii}=5,6,7,8,9, \rightarrow b_i=4\\
\end{array}
\right.
$
\\	\\
Da questa definizione risulta che $b_i \neq a_{ii} \forall i$. Osserviamo che per x $\in$ [0,1] (perchè del tipo $ 0,b_1 b_2 b_3 ...$ con i decimali tutti 4 o 5) e non è uguale a nessuno degli $r_i$ precedenti.
\\
\begin{tabular}{l}
$r \neq r_1$ perchè $b_1 \neq a_{11}$\\
$r \neq r_2$ perchè $b_2 \neq a_{22}$\\
$r \neq r_3$ perchè $b_3 \neq a_{33}$
\end{tabular}
\\
Questa è una contraddizione perchè avevamo supposto che gli $r_i$ esaurissero tutti i numeri reali dell'intervallo [0,1].
\\
\textbf{[0,1] non è numerabile, quindi R non è numerabile.}

\subsection{Prodotto in forma trigonometrica di numeri di $\emph{C}$}
\subsubsection*{Dimostrazione}
\begin{math}
z_1=\rho_1(\cos(\Theta_1)+i\sin(\Theta_1)),z_2=\rho_2(\cos(\Theta_2)+i\sin(\Theta_2))\\
z_1z_2=\rho_1\rho_2(\cos(\Theta_1)\cos(\Theta_2)-\sin(\Theta_1)\sin(\Theta_2)+i(\sin(\Theta_1)\cos(\Theta_2)+\cos(\Theta_1)\sin(\Theta_2))\\
z_1z_2=\rho_1\rho_2(\cos(\Theta_1+\Theta_2)+i\sin(\Theta_1+\Theta_2))
\end{math}

\subsection{Potenza e radice $\textit{n-esima}$ di un numero complesso}
\subsubsection*{Dimostrazione}
Sia w $\in$ C, $w\neq0$ e $n\geq 1$. Esistono esattamente n radici n-esime complesse $z_0,z_1, ... , z_{n-1}$ di w. Posto $ w=r(\cos\Theta + i\sin\Theta)$ e $z_k = \rho_k(\cos\Theta_k + i\sin\Theta_k)$ abbiamo:\\
$\rho_k=r^{\frac{1}{n}}, \hspace{0.2cm} \Theta_k=\frac{\rho+2k\Pi}{n}, \hspace{0.2cm} k=0,1,2,...,n-1	$\\
Mostriamo che non esistono altre radici. Un numero $R(\cos\gamma + i\sin\gamma)$ per essere radice n-esima di w, dovrebbe essere:\\
$R^n=r \hspace{0.2cm} e \hspace{0.2cm} n\gamma=\rho+2k\Pi \hspace{0.2cm} con \hspace{0.2cm} h\in Z$\\ cioè:\\ 
$R=r^{\frac{1}{n}} \hspace{0.2cm} e \hspace{0.2cm} \gamma=\frac{\rho}{n}+\frac{2h\Pi}{n}$
\\
\\
Dando ad h i valori 0,1,...,n-1 troviamo i numeri $z_k$. Dando un altro valore $\overline{h}=k+mn$ ($m\in Z$ è il quoziente e k è il resto della divisione di $\overline{h}$ per n), per cui: $\gamma=\frac{\rho}{n}+\frac{2k\Pi}{n}+2m\Pi=\Theta_k+2m\Pi$ ritroveremo gli stessi $z_k$ precedenti.\\\\
Abbiamo dimostrato che \textbf{Esistono esattamente n radice n-esime complesse.}

\newpage

\section{Successioni}
\subsection{Teorema di Monotonia}
\subsubsection*{Dimostrazione}
Sia $\left\{
\begin{array}{l}
a_n
\end{array}
\right\}
$ una successione monotona crescente e superiormente limitata. Allora $\left\{\begin{array}{l}a_n\end{array}\right\}$ è convergente e il suo limite è uguale a sup$\left\{\begin{array}{l}a_n : n \in N \end{array}\right\}$.\\
Consideriamo l'insieme dei valori assunti dalla successione, questo insieme è limitato superiormente quindi esiste il sup. $\Lambda=sup\left\{\begin{array}{l}a_n : n \in N \end{array}\right\}, \hspace{0.1cm} \Lambda \in R$.\\
Proviamo che $\lim_{n\to+\infty}a_n=\Lambda$.\\ Bisogna mostrare che $\forall\varepsilon>0, \hspace{0.2cm} \Lambda-\varepsilon<a_n<\Lambda+\varepsilon$ definitivamente.\\
$\forall\hspace{0.1cm} n, \hspace{0.2cm} a_n\leq\Lambda+\varepsilon$ definitivamente perchè $\Lambda$ è estremo superiore.\\
Per definizione $\Lambda$ è il minimo dei maggioranti. Perciò essendo $\Lambda-\varepsilon<\Lambda$ certamente, $\Lambda-\varepsilon$ non è un maggiorante, quindi $\exists\hspace{0.1cm}n_0 : a_{n_0}<\Lambda-\varepsilon$. Essendo monotona crescente $\forall\hspace{0.1cm}n\geq n_0, \hspace{0.1cm}\Longrightarrow\hspace{0.1cm}a_{n_0}>\Lambda-\varepsilon$ definitivamente.\\
\textbf{Quindi $\lim_{n\to+\infty}a_n=\Lambda$.}

\subsection{Corollario del Teorema di Monotonia}
\subsubsection*{Dimostrazione}
Sia $\left\{\begin{array}{l}a_n\end{array}\right\}$ una successione monotona crescente.\\ Allora esiste $\lim_{n\to+\infty}a_n=sup\left\{\begin{array}{l}a_n : n \in N \end{array}\right\}$. Se $\left\{\begin{array}{l}a_n\end{array}\right\}$ è superiormente limitata allora converge, se $\left\{\begin{array}{l}a_n\end{array}\right\}$ è superiormente illimitata allora diverge a $+\infty$.\\
Se la successione è superiormente limitata abbiamo il Teorema di Monotonia.\\
Se $\left\{\begin{array}{l}a_n\end{array}\right\}$ è illimitata superiormente, vuol dire che $\forall\hspace{0.1cm}k>0, \exists\hspace{0.1cm} n_0 : a_{n_0}>k$ definitivamente.\\ Essendo la successione crescente ogni $n>n_0 \Longrightarrow a_n \geq a_{n_0}>k$. Abbiamo provato che $\forall\hspace{0.1cm}k>0, a_n>k$ definitivamente.\\
\textbf{Quindi $a_n \rightarrow +\infty$}

\subsection{Teorema di unicità del limite}
\subsubsection*{Dimostrazione}
Se esiste $\lim_{x\to c}f(x)=l$, tale limite l è unico. Infatti, se esistessero due limiti $l_1 e l_2$ diversi tra lor, presa una qualsiasi successione $x_n\rightarrow c$ si avrebbe: $f(x_n)\rightarrow l_1$ e $f(x_2)\rightarrow l_2$ dunque la successione $f(x_n)$ avrebbe due limiti distinti, assurdo.\\
\textbf{Il Limite è quindi unico.}

\subsection{Algebra dei Limiti (somma e prodotto)}
\subsubsection*{Dimostrazione Somma}
$a_n \rightarrow a, \hspace{0.2cm} b_n\rightarrow b \Longrightarrow a_n+b_n \rightarrow a+b$\\
Fissiamo $\varepsilon > 0$ e consideriamo:\\
$|(a_n + b_n)-(a+b)|\leq |a_n-a|+|b_n-b|$\\
$|a_n-a|<\varepsilon \hspace{0.2cm}$ e $\hspace{0.2cm}|b_n-b|<\varepsilon$ definitivamente\\
$|(a_n+b_n)-(a+b)|<2\varepsilon$ definitivamente\\
$\Longrightarrow a_n+b_n \rightarrow a+b$
\subsubsection*{Dimostrazione Prodotto}
$a_n \rightarrow a, \hspace{0.2cm} b_n \rightarrow b \Longrightarrow a_nb_n\rightarrow ab$\\
Fissiamo $\varepsilon > 0$:\\
$|a_nb_n-ab|=|a_n(b_n+b)+b(a_n-a)|\leq|a_n(b_n-b)|+|b(a_n-a)|=|a_n||b_n-b|+|b||a_n-a|$\\
Poichè $a_n \rightarrow a, \hspace{0.2cm} |a_n-a|<\varepsilon$ definitivamente e inoltre $|a_n|<|a|+\varepsilon$ definitivamente.\\
Poichè $b_n \rightarrow b, \hspace{0.2cm} |b_n-b|<\varepsilon$ definitivamente.\\
Quindi: $|a_nb_n-ab|<(|a|+\varepsilon)\varepsilon+|b|\varepsilon<\varepsilon*costante$ definitivamente.\\
Per l'arbitrarietà di $\varepsilon$ segue la tesi.\\
$\Longrightarrow a_nb_n \rightarrow ab$

\subsection{Teorema di permanenza del segno}
\subsubsection*{Dimostrazione}
Se $a_n\rightarrow a$ e $a>0$ allora $a_n>0$ definitivamente.\\
Fissato $\varepsilon>0$, per definizione di limite abbiamo: $|a_n-a|<\varepsilon$ definitivamente.\\
$a-\varepsilon<a_n<a+\varepsilon$ definitivamente. Poichè $a>0$ possiamo scegliere $\varepsilon>0$ in modo che $a-\varepsilon>0$ allora $a-\varepsilon<a_n$.\\
\textbf{Mostro che $a_n>0$ definitivamente.}

\subsection{Teorema del confronto}
\subsubsection*{Dimostrazione}
Se $a_n \leq b_n \leq c_n$ definitivamente e $a_n \rightarrow l$ e $c_n \rightarrow l \in R$ allora anche $b_n \rightarrow l$.\\
Fissiamo $\varepsilon>0$, allora definitivamente abbiamo: \\$l-\varepsilon<a_n<l+\varepsilon$; $\hspace{0.5cm} l-\varepsilon<c_n<l+\varepsilon$\\ Segue definitivamente $l-\varepsilon<a_n \leq b_n \leq c_n<l+\varepsilon$\\ Quindi definitivamente $l-\varepsilon<b_n<l+\varepsilon$\\
\textbf{$\Longrightarrow b_n\rightarrow l$.}

\subsection{Limiti notevoli di successioni}
\subsubsection*{Dimostrazione}

\newpage

\section{Serie Numeriche}
\subsection{Carattere della $\emph{Serie Geometrica}$ e della $\emph{Serie di Mengoli}$}
\subsubsection*{Dimostrazione}


\subsection{Condizione Necessaria per la convergenza di una serie}
\subsubsection*{Dimostrazione}


\subsection{Criterio del $\emph{Confronto}$}
\subsubsection*{Dimostrazione}


\subsection{Criterio del $\emph{Confronto Asintotico}$}
\subsubsection*{Dimostrazione}


\subsection{Criterio del $\emph{Rapporto}$}
\subsubsection*{Dimostrazione}


\subsection{Comportamento di $\sum\frac{1}{n}$}
\subsubsection*{Dimostrazione}


\subsection{Comportamento di $\sum\frac{1}{(n^\alpha)}$}
\subsubsection*{Dimostrazione}


\subsection{Comportamento di $\sum\frac{1}{(n^\alpha\ln^\beta n)}$}
\subsubsection*{Dimostrazione}

\newpage

\section{Funzioni}
\subsection{Teorema della continuità della Funzione Composta}
\subsubsection*{Dimostrazione}


\subsection{Teorema degli Zeri}
\subsubsection*{Dimostrazione}


\subsection{Teorema di Darboux o dei valori intermedi}
\subsubsection*{Dimostrazione}


\subsection{Teorema di continuità della funzione inversa}
\subsubsection*{Dimostrazione}

\end{document}
