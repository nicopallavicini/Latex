\documentclass[11pt, a4paper]{book}
\author{Nicolò Pallavicini}
\title{Analisi Matematica 1}
\usepackage[italian]{babel}
\begin{document}

\maketitle

\chapter{Insiemi Numerici}
\section{Irrazionalità di $\sqrt{2}$}
\section{Numerabilità di $\emph{Q}$}
\section{Non Numerabilità di $\emph{R}$}
\section{Prodotto in forma trigonometrica di numeri di $\emph{C}$}
\section{Potenza e radice $\textit{n-esima}$ di un numero complesso}

\newpage
\begin{flushleft}
\section*{1.1}

\end{flushleft}




\chapter{Successioni}
\section{Teorema di Monotonia}
\section{Corollario del Teorema di Monotonia}
\section{Teorema di unicità del limite}
\section{Algebra dei Limiti (somma e prodotto)}
\section{Teorema di permanenza del segno}
\section{Teorema del confronto}
\section{Limiti notevoli di successioni}

\chapter{Serie Numeriche}
\section{Carattere della $\emph{Serie Geometrica}$ e della $\emph{Serie di Mengoli}$}
\section{Condizione Necessaria per la convergenza di una serie}
\section{Criterio del $\emph{Confronto}$}
\section{Criterio del $\emph{Confronto Asintotico}$}
\section{Criterio del $\emph{Rapporto}$}
\section{Comportamento di $\sum\frac{1}{n}$}
\section{Comportamento di $\sum\frac{1}{(n^\alpha)}$}
\section{Comportamento di $\sum\frac{1}{(n^\alpha\ln^\beta n)}$}

\chapter{Funzioni}
\section{Teorema della continuità della Funzione Composta}
\section{Teorema degli Zeri}
\section{Teorema di Darboux o dei valori intermedi}
\section{Teorema di continuità della funzione inversa}


\end{document}